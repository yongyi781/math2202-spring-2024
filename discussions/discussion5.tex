\documentclass[11pt,oneside]{amsart}
\usepackage{geometry}
\usepackage{amssymb,mathtools,microtype,version,pgfplots,booktabs}
\usepackage[shortlabels]{enumitem}
\usepackage[colorlinks]{hyperref}
\usepackage[most]{tcolorbox}
\pgfplotsset{compat=1.18}
\usepgfplotslibrary{fillbetween}

% Blackboard bold
\newcommand{\bA}{\mathbb A}
\newcommand{\bB}{\mathbb B}
\newcommand{\bC}{\mathbb C}
\newcommand{\bD}{\mathbb D}
\newcommand{\bE}{\mathbb E}
\newcommand{\bF}{\mathbb F}
\newcommand{\bG}{\mathbb G}
\newcommand{\bH}{\mathbb H}
\newcommand{\bI}{\mathbb I}
\newcommand{\bJ}{\mathbb J}
\newcommand{\bK}{\mathbb K}
\newcommand{\bL}{\mathbb L}
\newcommand{\bM}{\mathbb M}
\newcommand{\bN}{\mathbb N}
\newcommand{\bO}{\mathbb O}
\newcommand{\bP}{\mathbb P}
\newcommand{\bQ}{\mathbb Q}
\newcommand{\bR}{\mathbb R}
\newcommand{\bS}{\mathbb S}
\newcommand{\bT}{\mathbb T}
\newcommand{\bU}{\mathbb U}
\newcommand{\bV}{\mathbb V}
\newcommand{\bW}{\mathbb W}
\newcommand{\bX}{\mathbb X}
\newcommand{\bY}{\mathbb Y}
\newcommand{\bZ}{\mathbb Z}
\newcommand{\Fq}{\bF_q}
\newcommand{\Ga}{\bG_a}
\newcommand{\Gm}{\bG_m}

% Bold
\newcommand{\BA}{\mathbf A}
\newcommand{\BB}{\mathbf B}
\newcommand{\BC}{\mathbf C}
\newcommand{\BD}{\mathbf D}
\newcommand{\BE}{\mathbf E}
\newcommand{\BF}{\mathbf F}
\newcommand{\BG}{\mathbf G}
\newcommand{\BH}{\mathbf H}
\newcommand{\BI}{\mathbf I}
\newcommand{\BJ}{\mathbf J}
\newcommand{\BK}{\mathbf K}
\newcommand{\BL}{\mathbf L}
\newcommand{\BM}{\mathbf M}
\newcommand{\BN}{\mathbf N}
\newcommand{\BO}{\mathbf O}
\newcommand{\BP}{\mathbf P}
\newcommand{\BQ}{\mathbf Q}
\newcommand{\BR}{\mathbf R}
\newcommand{\BS}{\mathbf S}
\newcommand{\BT}{\mathbf T}
\newcommand{\BU}{\mathbf U}
\newcommand{\BV}{\mathbf V}
\newcommand{\BW}{\mathbf W}
\newcommand{\BX}{\mathbf X}
\newcommand{\BY}{\mathbf Y}
\newcommand{\BZ}{\mathbf Z}

% Calligraphic
\newcommand{\cA}{\mathcal A}
\newcommand{\cB}{\mathcal B}
\newcommand{\cC}{\mathcal C}
\newcommand{\cD}{\mathcal D}
\newcommand{\cE}{\mathcal E}
\newcommand{\cF}{\mathcal F}
\newcommand{\cG}{\mathcal G}
\newcommand{\cH}{\mathcal H}
\newcommand{\cI}{\mathcal I}
\newcommand{\cJ}{\mathcal J}
\newcommand{\cK}{\mathcal K}
\newcommand{\cL}{\mathcal L}
\newcommand{\cM}{\mathcal M}
\newcommand{\cN}{\mathcal N}
\newcommand{\cO}{\mathcal O}
\newcommand{\cP}{\mathcal P}
\newcommand{\cQ}{\mathcal Q}
\newcommand{\cR}{\mathcal R}
\newcommand{\cS}{\mathcal S}
\newcommand{\cT}{\mathcal T}
\newcommand{\cU}{\mathcal U}
\newcommand{\cV}{\mathcal V}
\newcommand{\cW}{\mathcal W}
\newcommand{\cX}{\mathcal X}
\newcommand{\cY}{\mathcal Y}
\newcommand{\cZ}{\mathcal Z}
\newcommand{\ck}{\mathcal k}

% Sans-serif
\newcommand{\sA}{\mathsf A}
\newcommand{\sB}{\mathsf B}
\newcommand{\sC}{\mathsf C}
\newcommand{\sD}{\mathsf D}
\newcommand{\sE}{\mathsf E}
\newcommand{\sF}{\mathsf F}
\newcommand{\sG}{\mathsf G}
\newcommand{\sH}{\mathsf H}
\newcommand{\sI}{\mathsf I}
\newcommand{\sJ}{\mathsf J}
\newcommand{\sK}{\mathsf K}
\newcommand{\sL}{\mathsf L}
\newcommand{\sM}{\mathsf M}
\newcommand{\sN}{\mathsf N}
\newcommand{\sO}{\mathsf O}
\newcommand{\sP}{\mathsf P}
\newcommand{\sQ}{\mathsf Q}
\newcommand{\sR}{\mathsf R}
\newcommand{\sS}{\mathsf S}
\newcommand{\sT}{\mathsf T}
\newcommand{\sU}{\mathsf U}
\newcommand{\sV}{\mathsf V}
\newcommand{\sW}{\mathsf W}
\newcommand{\sX}{\mathsf X}
\newcommand{\sY}{\mathsf Y}
\newcommand{\sZ}{\mathsf Z}

% Bold lowercase
\newcommand{\ba}{\mathbf a}
\newcommand{\bb}{\mathbf b}
\newcommand{\bc}{\mathbf c}
\newcommand{\bd}{\mathbf d}
\newcommand{\be}{\mathbf e}
\newcommand{\bff}{\mathbf f}
\newcommand{\bg}{\mathbf g}
\newcommand{\bh}{\mathbf h}
\newcommand{\bi}{\mathbf i}
\newcommand{\bj}{\mathbf j}
\newcommand{\bk}{\mathbf k}
\newcommand{\bl}{\mathbf l}
\newcommand{\bm}{\mathbf m}
\newcommand{\bn}{\mathbf n}
\newcommand{\bo}{\mathbf o}
\newcommand{\bp}{\mathbf p}
\newcommand{\bq}{\mathbf q}
\newcommand{\br}{\mathbf r}
\newcommand{\bs}{\mathbf s}
\newcommand{\bt}{\mathbf t}
\newcommand{\bu}{\mathbf u}
\newcommand{\bv}{\mathbf v}
\newcommand{\bw}{\mathbf w}
\newcommand{\bx}{\mathbf x}
\newcommand{\by}{\mathbf y}
\newcommand{\bz}{\mathbf z}

% Fraktur lowercase
\newcommand{\fa}{\mathfrak a}
\newcommand{\fb}{\mathfrak b}
\newcommand{\fc}{\mathfrak c}
\newcommand{\fd}{\mathfrak d}
\newcommand{\fe}{\mathfrak e}
\newcommand{\ff}{\mathfrak f}
\newcommand{\fg}{\mathfrak g}
\newcommand{\fh}{\mathfrak h}
% \fi already defined
\newcommand{\ffi}{\mathfrak i}
\newcommand{\fj}{\mathfrak j}
\newcommand{\fk}{\mathfrak k}
\newcommand{\fl}{\mathfrak l}
\newcommand{\fm}{\mathfrak m}
\newcommand{\fn}{\mathfrak n}
\newcommand{\fo}{\mathfrak o}
\newcommand{\fp}{\mathfrak p}
\newcommand{\fq}{\mathfrak q}
\newcommand{\fr}{\mathfrak r}
\newcommand{\fs}{\mathfrak s}
\newcommand{\ft}{\mathfrak t}
\newcommand{\fu}{\mathfrak u}
\newcommand{\fv}{\mathfrak v}
\newcommand{\fw}{\mathfrak w}
\newcommand{\fx}{\mathfrak x}
\newcommand{\fy}{\mathfrak y}
\newcommand{\fz}{\mathfrak z}

% The most common variants of single letters
\newcommand{\A}{\bA}
\newcommand{\B}{\cB}
\newcommand{\C}{\cC}
\newcommand{\D}{\cD}
\newcommand{\E}{\cE}
\newcommand{\F}{\cF}
\newcommand{\G}{\cG}
\newcommand{\I}{\cI}
\newcommand{\J}{\cJ}
\newcommand{\M}{\cM}
\newcommand{\N}{\bN}
\newcommand{\Q}{\bQ}
\newcommand{\R}{\bR}
\newcommand{\T}{\cT}
\newcommand{\U}{\cU}
\newcommand{\V}{\cV}
\newcommand{\W}{\cW}
\newcommand{\X}{\cX}
\newcommand{\Y}{\cY}
\newcommand{\Z}{\bZ}
\newcommand{\g}{\fg}
\newcommand{\h}{\fh}

\newcommand{\eps}{\varepsilon}

\DeclareMathOperator{\Var}{Var}
\let\Re\relax
\DeclareMathOperator{\Re}{Re}
\let\Im\relax
\DeclareMathOperator{\Im}{Im}
\DeclareMathOperator{\Res}{Res}
\DeclareMathOperator{\ord}{ord}
\DeclareMathOperator{\dir}{\mathbf{dir}}
\DeclareMathOperator{\divv}{div}
\DeclareMathOperator{\curl}{\mathbf{curl}}

\definecolor{sol}{rgb}{0.1, 0.3, 0.6}
\definecolor{pracsol}{rgb}{0.1, 0.6, 0.3}

\newtcolorbox{solution}{enhanced, breakable, colframe=sol, title=Solution}

\newtcolorbox{pracsol}{enhanced, breakable, colframe=pracsol, title=Practice Solution}

\theoremstyle{definition}
\newtheorem{problem}{Problem}
\newtheorem{question}{Question}
\newtheorem{practice}{Practice}
\newtheorem*{hint}{Hint}

\theoremstyle{plain}
\newtheorem{theorem}{Theorem}


\title{MATH2202 Spring 2024\\
Discussion 5}
\date{February 20, 2024}

\theoremstyle{definition}
\newtheorem{question}{Question}

\begin{document}
  \maketitle
First, TA will review the first two homework problems.
  \begin{problem}[Pset3-problem 4b]
   The functions
  \[f(x_1,x_2,\dots,x_{10000})=x_1x_2\cdots x_{10000}\]
  and
  \[g(x_1,x_2,\dots,x_{10000})=x_{10000}!\]
  are both functions in 10000 variables. Here, the notation $n!$ means the product of the positive integers from 1 to $n$ inclusive.
  \begin{enumerate}[(b)]
    
    \item Recall that factorial is defined only for non-negative integer inputs $0,1,2,\dots$. Given this information, what is the natural domain of $g$, in other words, the largest subset of $\bR^{10000}$ on which $g$ is defined?
    \begin{solution}
      $g(x_1,\dots,x_{10000})$ is defined as long as $x_{10000}$ is a non-negative integer. Therefore the domain of $g$ is $\{(x_1,\dots,x_{10000})\mid x_{10000}\geq 0\text{ and }x\text{ is an integer}\}$. There are plenty of equivalent correct formulations:
      \begin{itemize}
        \item $\{(x_1,\dots,x_{10000})\mid x_{10000}\text{ is a non-negative integer}\}$
        \item $\{(x_1,\dots,x_{10000})\mid x_{10000}\in\bZ\text{ and }x_{10000}\geq 0\}$
        \item $\{(x_1,\dots,x_{10000})\mid x_{10000}\in\bZ_{\geq 0}\}$
      \end{itemize}
      The following are common incorrect answers:
      \begin{itemize}
        \item $\{(x_1,\dots,x_{10000})\mid {\color{red}x_{10000}\geq 0}\}$ (missed the condition that $x_{10000}$ must be an integer)
        \item $\{(x_1,\dots,x_{10000})\mid {\color{red}\text{all }x_i\text{ are non-negative integers}}\}$ (missed that $x_1,\dots,x_{9999}$ is allowed take on non-integral values since $g$ is only taking the factorial of $x_{10000}$)
        \item {\color{red}$[0,\infty)$} (the main issue is that this describes a set of scalars, rather than a set of 10000-tuples (vectors with 10000 components). It also suffers from the above two issues)
      \end{itemize}

      7 out of 43 students answered this question completely correctly, so take care to read this solution 15 times. To assist in understanding, I will give more examples of functions and their domains below:
      \begin{itemize}
        \item The function $f(x)=x!$ has domain $\bZ_{\geq 0}$, the set of non-negative integers.
        \item The function $f(x,y)=y\cdot x!$ has domain $\{(x,y)\mid x\in\bZ_{\geq 0}\text{ and }y\in\bR\}$.
        \item The function $f(x,y,z)=(x^2)!$ has domain $\{(\pm\sqrt n,y,z)\mid n\in\bZ_{\geq 0}\text{ and }y,z\in\bR\}$. This can also be written as $\{(x,y,z)\mid y,z\in\bR\text{ and }x=\pm\sqrt n\text{ for some }n\in\bZ_{\geq 0}\}$.
        \item The function $f(x_1,x_2,\dots,x_{10000})=1/x_1$ has domain $\{(x_1,\dots,x_{10000})\mid x_1\neq 0\}$.
        \item The function $f(x,y)=\log(x-y)$ has domain $\{(x,y)\mid x>y\}$.
      \end{itemize}
    \end{solution}
  
  \end{enumerate}
\end{problem}


\begin{problem}[Pset3-problem 2a]\leavevmode
  \begin{enumerate}[(a)]
    \item Show that the curves $t\mapsto (t^3+t^2-3t+1,t^2,3t-1)$ and $t\mapsto(3t+1,2t,t^2+t-1)$ intersect at two points.
    \begin{solution}
      The system of equations to solve is:
      \begin{align*}
        t^3+t^2-3t+1 &= 3s+1\\
        t^2 &= 2s\\
        3t-1 &= s^2+s-1.
      \end{align*}
      Note that we need to use two independent variables because it is not necessarily the case that the points of intersection correspond to the same parameter $t$ for both curves. Indeed, the paths of two particles can cross while the particles themselves miss each other. (See p.774--775 for an example in the textbook about this.)

      We can take the second equation, rewrite it as $s=\frac{t^2}2$, and make that substitution into the first and third equation. Now we have two equations in $t$:
      \begin{align*}
        t^3+t^2-3t+1 &= \frac32t^2+1\\
        3t-1 &= \frac94t^4+\frac32t^2-1.
      \end{align*}
      Let's solve the first equation first, since is a polynomial equation of degree 3 while the other one has degree 4. This first equation simplifies to
      \[\begin{split}
        t^3-\frac12t^2-3t=0 &\iff t\left(t^2-\frac12t-3\right)=0\\
        &\iff t(2t^2-t-6)=0\\
        &\iff t(2t+3)(t-2)=0.
      \end{split}\]
      Therefore the first equation has solutions $t=0,-\frac32,2$. Out of these, the solutions $t=0$ and $t=2$ also satisfy the last equation $3t-1=\frac94t^4+\frac32t^2-1$.

      At $t=0$, the corresponding point is $(1,0,-1)$, and at $t=2$ the corresponding point is $(7,4,5)$. Thus, the two curves intersect at two points.

      \textbf{Remark}: This problem has the coincidence that the incorrect solution that uses 1 variable $t$ in both sides of the system of equations still gets the correct answer. This does not mean it is a valid solution, though.
    \end{solution}
   
  \end{enumerate}
\end{problem}

Next, you can discuss in groups on the following homework problems:
\begin{problem}[Pset1-problem 3(b)]
  The triangle inequality says that for all vectors $\bv$ and $\bw$,
  \[\|\bv+\bw\|\leq \|\bv\|+\|\bw\|.\]
  Of course I haven't proven it, but in part (b) of this problem you will prove it.
    \begin{enumerate}[(b)]
    
      
      \item Using the Cauchy-Schwarz inequality, prove the triangle inequality.
      \begin{solution}
        Let us prove the equivalent inequality $\|\bv+\bw\|^2\leq(\|\bv\|+\|\bw\|)^2$ instead. (Equivalent because both sides of the origial inequality are non-negative.) We have
        \[\begin{split}
          \|\bv+\bw\|^2 &= (\bv+\bw)\cdot(\bv+\bw)\\
          &= \bv\cdot\bv+2\bv\cdot\bw+\bw\cdot\bw\\
          &= \|\bv\|^2+2\bv\cdot\bw+\|\bw\|^2\\
          &\leq\|\bv\|^2+2\|\bv\|\|\bw\|+|\bw\|^2\qquad\text{(by Cauchy-Schwarz)} \\
          &=(\|\bv\|+\|\bw\|)^2.
        \end{split}\]
     
      \end{solution}
  \end{enumerate}
\end{problem}
\begin{problem}[Pset2-Problem4]
  Find a formula for the distance between the two planes with Cartesian equations $Ax+By+Cz=D_1$ and $Ax+By+Cz=D_2$. You can use the following fact: if $P$ and $Q$ are two points on the two respective planes with minimum possible distance, then the vector $\overrightarrow{PQ}$ is perpendicular to both planes.

\end{problem}
\begin{solution}
  Let $P=(x,y,z)$ and $Q$ be as in the hint: $P$ is on the first plane, $Q$ is on the second plane, and $Q$ is chosen so that $\|\overrightarrow{PQ}\|$ is minimal.

  According to the fact in the problem, $\overrightarrow{PQ}\parallel\bn$, where $\bn=(A,B,C)$ is the common normal vector to both planes. (The fact that $\bn=(A,B,C)$ is because those are the coefficients of $x,y,z$ in both planes.)

  So let $\overrightarrow{PQ}=\lambda\bn$. Moreover, we have the following:
  \begin{align*}
    P\cdot\bn&= D_1\\
    Q\cdot\bn &= D_2,
  \end{align*}
  because, for example, the expression $P\cdot\bn$ equals $(x,y,z)\cdot(A,B,C)=Ax+By+Cz$ which is exactly equal to $D_1$ because of the fact that $P$ lies on the first plane. Similar reasoning holds for $Q$. Therefore,
  \[\overrightarrow{PQ}\cdot\bn=(Q-P)\cdot \bn=Q\cdot \bn-P\cdot \bn=D_2-D_1,\]
  But $\overrightarrow{PQ}\cdot\bn=(\lambda\bn)\cdot\bn=\lambda\|\bn\|^2$. Therefore, $\lambda\|\bn\|^2=D_2-D_1$, so $\lambda=\frac{D_2-D_1}{\|\bn\|^2}$. Finally,
  \[\|\overrightarrow{PQ}\|=\|\lambda\bn\|=|\lambda|\|\bn\|=\frac{|D_2-D_1|}{\|\bn\|^2}\|\bn\|=\frac{|D_2-D_1|}{\sqrt{A^2+B^2+C^2}}.\]
\end{solution}

  
\end{document}
