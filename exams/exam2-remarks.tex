\documentclass[11pt,oneside]{amsart}
\usepackage[margin=1in]{geometry}
\usepackage{amssymb,parskip,mathtools,microtype,pgfplots}
\usepackage[shortlabels]{enumitem}
\usepgfplotslibrary{fillbetween}

\theoremstyle{definition}
\newtheorem{problem}{Problem}
\newtheorem{question}{Question}
\newtheorem*{remark}{Remark}

\theoremstyle{plain}
\newtheorem{theorem}{Theorem}

\newcommand{\bC}{\mathbb{C}}
\newcommand{\bQ}{\mathbb{Q}}
\newcommand{\bR}{\mathbb{R}}
\newcommand{\bZ}{\mathbb{Z}}
\newcommand{\bE}{\mathbb{E}}
\newcommand{\BP}{{\mathbf{P}}}
\newcommand{\ba}{{\mathbf{a}}}
\newcommand{\bi}{{\mathbf{i}}}
\newcommand{\bj}{{\mathbf{j}}}
\newcommand{\bk}{{\mathbf{k}}}
\newcommand{\bn}{{\mathbf{n}}}
\newcommand{\br}{{\mathbf{r}}}
\newcommand{\bs}{{\mathbf{s}}}
\newcommand{\bu}{{\mathbf{u}}}
\newcommand{\bv}{{\mathbf{v}}}
\newcommand{\bw}{{\mathbf{w}}}
\newcommand{\eps}{\varepsilon}
\newcommand{\blank}{\underline{\hspace{1cm}}}
\newcommand{\longblank}{\underline{\hspace{2cm}}}

\DeclareMathOperator{\Var}{Var}
\let\Re\relax
\DeclareMathOperator{\Re}{Re}
\let\Im\relax
\DeclareMathOperator{\Im}{Im}
\DeclareMathOperator{\Res}{Res}
\DeclareMathOperator{\ord}{ord}
\DeclareMathOperator{\dir}{\mathbf{dir}}

\title{MATH2202 Spring 2024\\
Exam 2}
\author{Wednesday, April 10, 2024}

\begin{document}
\maketitle

Name: \underline{\hspace{6cm}}

This exam is open book. Calculators are not allowed. There are 50 points total in this exam. If you do not manage to solve a problem, show a strategy you tried and a reflection on why it did not work, for partial credit.

\vskip 2cm

Please answer the following questions. {\color{blue} The boxed answers had the best averages.}

\begin{question}
  I did the practice problems on the homework. (Circle one.)

  \hspace{1.5cm}\boxed{\text{Yes, all of them}}\hspace{1.5cm} Some of them\hspace{1.5cm} No
\end{question}

\begin{question}
  I did the midterm additional practice problems. (Circle one.)

  \hspace{1.5cm}\boxed{\text{Yes, all of them}}\hspace{1.5cm} \boxed{\text{Some of them}}\hspace{1.5cm} No\hspace{1.5cm}\boxed{\text{What's that?}}
\end{question}

\begin{question}
  I understood the homework solutions. (Circle one.)

  \hspace{1.5cm}\boxed{\text{Yes, all of them}}\hspace{1.5cm} Some of them\hspace{1.5cm} No
\end{question}

\begin{question}
  I can understand the textbook. (Circle one.)

  \hspace{1.5cm}\boxed{\text{Yes}}\hspace{1.5cm} \boxed{\text{Only the examples}}\hspace{1.5cm} No, but I tried\hspace{1.5cm}Didn't read
\end{question}

\begin{question}
  I come to class. (Circle one.)

  \hspace{1.5cm}\boxed{\text{Yes and I mainly listen}}\hspace{1.5cm}\boxed{\text{Yes and I take notes}}\hspace{1.5cm} No
\end{question}

\begin{question}
  I come to discussions. (Circle one.)

  \hspace{1.5cm}\boxed{\text{Yes}}\hspace{1.5cm} No
\end{question}

\begin{question}
  I studied using: (Circle all that apply.)

  \hspace{1.5cm}\boxed{\text{Homework solutions}}\hspace{0.05\textwidth} \boxed{\text{Practice problems}}\hspace{0.05\textwidth} Materials from previous years

  \hspace{1.5cm}Other (please specify): \underline{\hspace{8cm}}
\end{question}

\newpage

\begin{problem}\leavevmode
  \begin{enumerate}[(a)]
    \item (5 points) The point $(\rho,\theta,\phi)=(3,\pi/2,\pi/2)$ is given in spherical coordinates. What is it in rectangular coordinates?
    \begin{remark}\color{blue}
      The answer is $(0,3,0)$. On the exam, I mistakenly had $r$ instead of $\rho$. However, the final answer would be the same anyway.
    \end{remark}
    \vfill
    \item (5 points) Swap the order of integration of the following double integral, but do not evaluate.
    \[\int_1^4\int_{y}^{y^2}\frac 1y\,dx\,dy.\]
    Your answer should be of the form
    \[\int_{\boxed{?}}^{\boxed{?}}\int_{\boxed{?}}^{\boxed{?}}\boxed{?}\,dy\,dx+\int_{\boxed{?}}^{\boxed{?}}\int_{\boxed{?}}^{\boxed{?}}\boxed{?}\,dy\,dx.\]
    Drawing a picture is highly recommended.
    \begin{remark}\color{blue}
      Here is a plot of the region of integration.
      \begin{center}
        \begin{tikzpicture}
          \begin{axis}[axis lines=middle, axis equal, xmin=0, xmax=16, ymin=0, ymax=4, xlabel=$x$, ylabel=$y$]
            \addplot[name path=A, domain=1:4, samples=50] (x,x);
            \addplot[name path=B, domain=1:16] (x,{sqrt(x)});
            \addplot[domain=4:16] (x,4);
            \addplot[blue!50, opacity=0.5] fill between[of=A and B];
    %         \fill[gray,opacity=0.2] plot[domain=0:1] (-2*\x-1,\x) -- plot[domain=-1:3]
    % (-\x*\x+2,\x);
          \end{axis}
        \end{tikzpicture}
      \end{center}
      This was difficult for many people. Not many people visualized/drew the region correctly. The answer is
      \[\int_1^4\int_{\sqrt x}^x\frac 1y\,dy\,dx+\int_4^{16}\int_{\sqrt x}^4\frac 1y\,dy\,dx.\]
      This problem is similar to Problem Set 7, Practice 4 and Problem 4.

      On copy of the exam that I administered, I accidentally omitted an integral from each of the terms in the desired form, but put a correction on the board. Nevertheless, the main source of difficulty of this problem was drawing/visualizing the region of integration accurately.
    \end{remark}
    \vfill
    \vfill
    \vfill
  \end{enumerate}
\end{problem}

\newpage

\begin{problem}\leavevmode
  \begin{enumerate}[(a)]
    \item (5 points) Compute the triple integral
    \[\iiint_{\mathcal U}dV\]
    where $\mathcal U=\{(x,y,z)\mid 0\leq y\leq 2,2\leq z\leq 4, y\leq x\leq z\}$.
    \begin{remark}\color{blue}
      The triple integral is
      \[\int_0^2\int_2^4\int_y^z dx\,dz\,dy.\]
      The answer is 8.
    \end{remark}
    \vfill
    \item (5 points) Write down (but do not solve) the Lagrange equations that would solve this problem: Find the point $(x,y,z)$ on the sphere of radius 6 centered at the origin closest to $(1,2,3)$.
    \begin{remark}\color{blue}
      We can use the objective function $f(x,y,z)=(x-1)^2+(y-2)^2+(z-3)^2$ and constraint $x^2+y^2=z^2=36$. Then the Lagrange equations are
      \begin{align*}
        x^2+y^2+z^2 &= 36\\
        2(x-1)&=\lambda\cdot 2x\\
        2(y-2)&=\lambda\cdot 2y\\
        2(z-3)&=\lambda\cdot 2z.
      \end{align*}
      There is also a more complicated variant if you used the objective function \\$\sqrt{(x-1)^2+(y-2)^2+(z-3)^2}$.
    \end{remark}
    \vfill
  \end{enumerate}
\end{problem}

\newpage

\begin{problem}
  Let $f(x,y)=xe^y+3x$.
  \begin{enumerate}[(a)]
    \item (3 points) Calculate $\nabla f(5,0)$.
    \begin{remark}\color{blue}
      The answer is $(4,5)$.
    \end{remark}
    \vfill
    \item (7 points) You're given that $\nabla f(3,0)=(4,3)$. Show that there is no unit vector $\bu$ such that $D_\bu f(3,0)=-6$.
    \begin{remark}\color{blue}
      The simplest solution is to notice that by the Cauchy-Schwarz inequality, $|\nabla f(3,0)\cdot \bu|\leq \|(4,3)\|\cdot1=5$, in other words, $-5\leq\nabla f(3,0)\cdot \bu\leq 5$. So $-6$ is not attainable as a directional derivative of $f$ at $(3,0)$, since $-6$ is outside the range $[-5,5]$.

      This is basically a copy of Problem Set 5, Problem 5. (Check out the solution for that problem. The above solution is the first solution.)

      Many people had more complicated solutions involving quadratics and discriminants, which would also work. (Corresponding to the third solution in Problem Set 5, Problem 5.)
    \end{remark}
    \vfill
    \vfill
    \vfill
  \end{enumerate}
\end{problem}

\newpage

\begin{problem}[10 points]
  The Archimiedean spiral $r=\theta$ is drawn below. Write down, but do not evaluate, an integral in polar coordinates that represents the shaded area. (Hint: Despite appearances, the arcs are not circular arcs.)
  \begin{center}
    \begin{tikzpicture}[scale=0.3]
      \draw[->] (0,-6*pi) -- (0,6*pi) node[above] {$y$};
      \draw[->] (-6*pi,0) -- (6*pi,0) node[right] {$x$};

      \draw[domain=0:6*pi,smooth,variable=\t,samples=100] plot ({\t*cos(deg(\t))},{\t*sin(deg(\t))});
      \fill[gray] (2*pi,0) -- plot[domain=4*pi:4.5*pi, variable=\t] ({\t*cos(deg(\t))},{\t*sin(deg(\t))}) -- (0,4*pi) -- plot[domain=2.5*pi:2*pi, variable=\t] ({\t*cos(deg(\t))},{\t*sin(deg(\t))}) -- cycle;
    \end{tikzpicture}
  \end{center}
\end{problem}
\begin{remark}\color{blue}
  Some possibilities (any of these would work):
  \[\int_0^{\frac{\pi}{2}}\int_{\theta+2\pi}^{\theta+4\pi}r\,dr\,d\theta.\]
  \[\int_{2\pi}^{\frac{5\pi}{2}}\int_{\theta}^{\theta+2\pi}r\,dr\,d\theta.\]
  \[\int_{4\pi}^{\frac{9\pi}2}\int_0^\theta r\,dr\,d\theta-\int_{2\pi}^{\frac{5\pi}2}\int_0^\theta r\,dr\,d\theta.\]
  \[\int_0^{\frac{\pi}2}\left(\frac12(\theta+4\pi)^2-\frac12(\theta+2\pi)^2\right)\,d\theta.\]
  I think this was the hardest problem on the exam. This had some of the same themes as one of the polar integration questions I put on the board in class just before spring break. It is also similar to Problem Set 7, Problem 3.

  In order to solve this problem, you needed to:
  \begin{itemize}
    \item Understand the graph of $r=\theta$ well enough to realize that the Archimedean spiral goes around the circle more than once.
    \item Identify the corner points of the shaded region as having distances $2\pi,4\pi,5\pi/2,9\pi/2$ from the origin.
    \item Identify the inner boundary curve as belonging to the $\theta\in [2\pi,5\pi/2]$ portion of the graph, and identify the outer boundary curve as belonging to the $\theta\in [4\pi,9\pi/2]$ portion of the graph.
    \item Be able to translate this knowledge using some thinking to an expression in terms of polar integration. (See the four integrals above.)
    \item Remember the $r$ factor as the Jacobian for polar integraion.
  \end{itemize}
  So this was definitely a multi-layered problem.

  Fun fact, the actual area (i.e.\ the result of any the integrals above) is $\dfrac{13\pi^3}4$.
\end{remark}

\newpage

\begin{problem}[10 points]
  Recall that a (twice-differentiable) function $f(x,y)$ is \emph{harmonic} if it satisfies
  \[\frac{\partial^2 f}{\partial x^2}(x,y)+\frac{\partial^2 f}{\partial y^2}(x,y)=0\text{ for all }x,y.\]
  Suppose $f(x,y)$ is a harmonic function and that for every critical point $(x_0,y_0)$ of $f$, the discriminant $\mathcal D(f,(x_0,y_0))$ is nonzero. Prove that $f$ has no local minima or maxima.\footnote{The claim is still true without the discriminant condition but would be harder to prove.}

  Hint: Start with the following sentence: ``Suppose that $P=(x_0,y_0)$ is a critical point of $f$.'' You may use the space below for scratch work and/or thinking.
\end{problem}
\begin{remark}\color{blue}
  People did well on this problem. Some key observations in order:
  \begin{itemize}
    \item The fact $f$ is harmonic means $f_{xx}(x,y)+f_{yy}(x,y)=0$ at each point $(x,y)$, in other words, $f_{xx}(x,y)=-f_{yy}(x,y)$.
    \item The product of two numbers with opposite signs is always negative or 0, because $x\cdot(-x)=-x^2\leq 0$.
    \item Hence $f_{xx}(x,y)f_{yy}(x,y)\leq 0$ for all $x,y$.
    \item Also, $f_{xy}(x,y)^2\geq 0$ for all $x,y$ because the left hand side is a square of a real number.
    \item Hence $\mathcal D(f,(x_0,y_0))\leq 0$ for all $(x_0,y_0)$.
    \item The hypotheses of the problem imply that the inequality is strict, so $(x_0,y_0)$ is a saddle point for all critical points $(x_0,y_0)$.
  \end{itemize}
\end{remark}

\end{document}
