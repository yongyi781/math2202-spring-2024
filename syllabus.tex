\documentclass[11pt,oneside]{amsart}
\usepackage{geometry,hyperref}

\pagestyle{empty}

\title{Course information for MATH2202 (Spring 2024)\\
    Multivariable Calculus}

\begin{document}
\maketitle

\noindent\textbf{Instructor:} Yongyi Chen\\
\textbf{Email:} \href{mailto:yongyi.chen@bc.edu}{\texttt{yongyi.chen@bc.edu}}

\noindent\textbf{Lectures:}
\begin{itemize}
    \item Section 1: MWF 10 am--10:50 am in Gasson Hall 205
    \item Section 2: MWF 9 am--9:50 am in Gasson Hall 205
\end{itemize}

\noindent\textbf{Homework:} Weekly, due on Wednesdays at 11:59 pm on Gradescope.

\noindent\textbf{Office:} Maloney 532

\noindent\textbf{Office hours:} (tentative) Mondays and Wednesdays at 11 am.

\section*{Course information}
\subsection*{Course website}
On Canvas. There you will find homework assignments, homework solutions, and supplemental course materials.

\subsection*{Course format}
In person.

\subsection*{Textbooks}
We will be following Blank and Krantz, \emph{Multivariable Calculus}, 2nd edition.

\subsection*{Contents}
\begin{itemize}
    \item Weeks 1 to 3: Vectors, 3-dimensional space, and vector functions (Chapters 9 and 10). Dot product, cross product, lines, planes, velocity, acceleration, tangent vectors, arc length.

    \item Weeks 3 to 8: Differentiation of multivariable functions (Chapter 11). Limits, continuity, partial derivatives, differentiability, chain rule, gradient, directional derivatives, tangent planes, maximum-minimum problems, Lagrange multipliers

    \item Weeks 9 to 12: Multiple integrals (Chapter 12). Double integral, integration in polar coordinates, triple integral, integration in cylindrical and spherical coordinates, change of coordinates.

    \item Weeks 13 to 15: Vector calculus (Chapter 13). Vector fields, divergence, gradient and curl, line integrals, Green's theorem, conservative vector fields and path-independence, surface integrals, Stokes' and divergence theorems.
\end{itemize}

\subsection*{Technology} We will use the computer software \emph{Mathematica} in this course. However, no prior computer knowledge is required. Also, we will not test you on any \emph{Mathematica} programming in the exams. The purpose of using \emph{Mathematica} is to help you visualize and understand mathematical concepts more easily.

\subsection*{Calculators} Calculators may be used anywhere in homework where they do not trivialize the problems. On exams, questions will be designed so that calculator use is unnecessary (and so they will not be allowed).

\subsection*{Homework}
Homework is the most important part of this course and is where the majority of your learning will occur. There will be weekly homework (also called problem sets), due on Wednesdays online on Gradescope at 11:59 pm. Late homework will not be accepted.

The first part of the homework will consist of practice problems from the textbook. You should do these, but do not turn them in.

The second part of the homework will consist of 3-5 more involved problems. These will be what you submit to Gradescope. To submit your homework, upload a single PDF file to Gradescope (accessible from within the Canvas assignment page as well), then select pages as instructed. Selecting pages is important as the grader will have to do extra work to locate your work each problem if you don't. Generally, you should write your solutions as if you were explaining them to a (very skeptical) classmate.

You are encouraged to collaborate on homework with your classmates. However, too much reliance on other people will cause you to learn less from the homework than intended. The recommended way to collaborate is to figure out the rough solution idea together, then write the details of the solutions on your own without assistance from the group.

If you've attempted a problem for a while but can't figure out how to solve it and it's too late to ask anyone, it's okay if you don't solve it -- these problems are challenging by nature, and an unsolved problem here or there will not mathematically impact your grade much if at all. Do submit your attempt, it is good to see that you attempted it. Moreover, if you have not solved a problem, you are expected to learn how it is solved (e.g. from reading the solutions).

Solutions will be provided soon after the due date for each problem set. I may use one of your solutions to a problem if I find it particularly elegant (which is another reason you should write your solutions as if you are explaining them to a skeptical classmate!)

2 bonus points (out of around 30) per homework assignment will be awarded to submissions done in \LaTeX, the mathematical typesetting language. Math looks very pretty in \LaTeX! See \url{https://www.overleaf.com/learn} to get started with \LaTeX. For example, this syllabus was typeset with \LaTeX.

The first assignment, Problem Set 0, will be more of a survey than a problem set. This assignment will be graded on completion.

\subsection*{Problem solving}
Problem solving is an integral part of mathematics, as well as life. Problem solving is how you get from not knowing how to solve a problem to figuring out a solution. On some days, instead of 50 minutes of lecture, we will have 30 minutes of lecture and have the last 20 minutes dedicated to problem solving workshop. Everyone is expected to participate in these problem solving workshops.

\subsection*{Discussion sections}
You are expected to attend both the lectures and discussion section. Discussion this semester is led by Tingting Fang. Discussion section is the prime time for you to get more practice, as well as to clear up any confusions you may have that you did not get to address during lectures.

\subsection*{Exams and grading}
The purpose of exams is to verify that you are doing your own work on the problem sets. Therefore, most exam problems will test you on ideas and insights gained from the homework.

There will be two in-class exams (50 minutes each) and a (cumulative) final (180 minutes). Final grades will be determined by a weighted average of homework and exam scores.  Homework counts for 20\%, each in-class exam counts for 20\%, and the final counts for 40\%.

All exams will be given in class, open book, but no calculators. Exam dates and times are as follows:
\begin{itemize}
  \item Exam 1: February 21, 2024, in class.
  \item Exam 2: April 10, 2024, in class.
  \item Final exam:
  \begin{itemize}
    \item Section 1 (MWF10): Wednesday, May 8, 2024 at 9:00 am.
    \item Section 2 (MWF9): Monday, May 13, 2024 at 9:00 am.
  \end{itemize}
\end{itemize}
Final note: In practice, if you ever want to know how you are doing in the class, it's more useful to think about your ability to solve problems instead of focusing on your numerical calculated grade at all times. It's not always about numbers!

\subsection*{Resources}
Besides my office hours and the TA's office hours, you can also get help from the following resources:
\begin{itemize}
    \item The math learning center in Maloney 536 -- drop in tutoring, no appointment necessary.
    \item The Connors Family Learning Center -- peer tutoring for all Boston College students.
\end{itemize}

\subsection*{Academic integrity}
Cheating of any kind will result in a zero for the relevant assignment or exam and referral to the Dean’s office for disciplinary action.  For more information on academic integrity see \url{https://www.bc.edu/integrity}.

\end{document}