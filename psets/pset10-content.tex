\maketitle

Due: Wednesday, April 24 at 11:59 pm.

\section{Practice problems -- do not submit}
\subsection{10.2. Line integrals}
The practice problems for line integrals are in the previous problem set.

\subsection{10.3. Conservative vector fields and path independence}
\begin{practice}p.1084 \#1\end{practice}
\begin{pracsol}
  This is a closed field on $Q$ because $M_y(x,y)=0=N_x(x,y)$ at all points $(x,y)$ in $Q$. Consequently, there is a function $(x,y)\mapsto u(x,y)$ on $Q$ such that $u_x(x,y)=x+\pi$ and $u_y(x,y)=1$.

  The first equation implies that $u(x,y)=\frac{x^2}{2}+\pi x+\phi(y)$ and the second equation will be satisfied if $\phi$ is defined so that $\phi'(y)=1$. That is, $\phi(y)=y+C$. All potential functions have the form
  \[V(x,y)=-u(x,y)-\frac{x^2}{2}-\pi x-y+C.\]
\end{pracsol}
\begin{practice}p.1084 \#7\end{practice}
\begin{pracsol}
  This is a closed field on $Q$ because $M_y(x,y)=0=N_x(x,y)$ at all points $(x,y)$ in $Q$. Consequently, there is a function $(x,y)\mapsto u(x,y)$ such that $u_x(x,y)=(x-2)^{-2}$ and $u_y(x,y)=\left(y-\frac12\right)^2$. That is, $\phi(y)=\frac{\left(y-\frac12\right)^3}{3}+C$. All potential functions have the form
  \[V(x,y)=-u(x,y)=(x-2)^{-1}-\frac{\left(y-\frac12\right)^3}{3}+C.\]
\end{pracsol}
\begin{practice}p.1084 \#13\end{practice}
\begin{pracsol}
  This is a closed field on $Q$. The required conditions are easily seen to be satisfied:
  \begin{align*}
    M_y &= z = N_x\\
    M_z &= y = R_x\\
    N_z &= x = R_y
  \end{align*}
  at all points $(x,y,z)$ in $Q$. Consequently, there is a function $(x,y,z)\mapsto u(x,y,z)$ on $Q$ such that $u_x(x,y,z)=yz+1$, $u_y(x,y,z)=xz+2$, and $u_z(x,y,z)=xy+3$.

  The first equation implies that $u(x,y,z)=xyz+x+\phi(y,z)$ and the second equation will be satisfied if $\phi$ is defined so that $xz+\phi_y(y,z)=xz+2$. That is, $\phi'(y)=2$, so $\phi(x,y)=2y+\psi(z)$ and $u(x,y,z)=xyz+x+2y+\psi(z)$. To satisfy the third equation choose $\psi$ so that $xy+\psi'(z)=xy+3$. That is, $\psi'(z)=3$, so $\psi(z)=3z+C$, a constant. All potential functions have the form
  \[V(x,y,z)=-u(x,y,z)=-xyz-x-2y-3z+C.\]
\end{pracsol}
\begin{practice}p.1084 \#31\end{practice}
\begin{pracsol}
  Using the following parametrizations:
  \[\br_1(t)=(0,t,1),\quad\br_2(t)=(t,1,1),\quad\text{and }\br_3(t)=(1-t,1-t,1),\]
  (each one for $0\leq t\leq 1$), the integrals over the three line segments evaluate as follows:
  \begin{align*}
    \int_{\overrightarrow{P_0P_1}}\BF\cdot d\br &= \int_0^1 ((t^2)\cdot 0+0\cdot 1+2\cdot 0)\,dt=0\\
    \int_{\overrightarrow{P_1P_2}}\BF\cdot d\br &= \int_0^1(2t\cdot 1+t^2\cdot 0+2\cdot 0)\,dt=1\\
    \int_{\overrightarrow{P_2P_0}}\BF\cdot d\br &= \int_0^1 (2(1-t)^2\cdot(-1)+(1-t)^2\cdot(-1)+2\cdot 0)\,dt\\
    &= \int_0^1 -3(1-t)^2\,dt\\
    &= (1-t)^3\Big|_0^1\\
    &= -1.
  \end{align*}
\end{pracsol}

\subsection{10.4. Divergence, gradient, and curl}
\begin{practice}p.1093 \#1\end{practice}
\begin{pracsol}
  $\nabla\cdot\BF=\divv(\BF)=\frac{\partial}{\partial x}(3x^2y^3)-\frac{\partial}{\partial y}(xy^4)=6xy^3-4xy^3=2xy^3$.
\end{pracsol}
\begin{practice}p.1093 \#13\end{practice}
\begin{pracsol}
  $\nabla\cdot\BF=\divv(\BF)=\frac{\partial}{\partial x}(x+2y+3z)+\frac{\partial}{\partial y}(2(x+2y+3z))+\frac{\partial}{\partial z}(3(x+2y+3z))=1+4+9=14$.
\end{pracsol}
\begin{practice}p.1093 \#23\end{practice}
\begin{pracsol}
  \[\begin{split}
    \nabla\times \BF &=\curl(\BF)\\
    &= \left(\frac{\partial}{\partial y}(xyz)-\frac{\partial}{\partial z}(-y^3)\right)\bi+\left(\frac{\partial}{\partial z}(zx)-\frac{\partial}{\partial x}(xyz)\right)\bj\\
    &\qquad\qquad+\left(\frac{\partial}{\partial x}(-y^3)-\frac{\partial}{\partial y}(zx)\right)\bk\\
    &= xz\bi+(x-yz)\bj.
  \end{split}\]
\end{pracsol}
\begin{practice}p.1093 \#33\end{practice}
\begin{pracsol}
  $\nabla\cdot(\nabla\times u)=\divv(\nabla u)=\divv(2x\bi-2y\bj)=2-2=0$.
\end{pracsol}
\begin{practice}p.1093 \#39\end{practice}
\begin{pracsol}
  $\nabla(\nabla\cdot\BF)=\nabla(\divv(\BF))=\nabla(-\sin(x)+\cos(y))=(-\cos(x),-\sin(y))$.
\end{pracsol}

\subsection{10.5. Green's theorem}
\begin{practice}p.1103 \#1\end{practice}
\begin{pracsol}
  The curve $\mathcal C$ is the circle $x^2+y^2=1$ traversed in the counter-clockwise direction. Since $\BF(\br(t))=(\cos^2(t),-3\sin(t))$ and $\br'(t)=(\cos(t),\sin(t))$, $\BF(\br(t))\cdot\br'(t)=\cos^3(t)-3\sin^2(t)$. Therefore,
  \[\begin{split}
    \oint_{\mathcal C}\BF\cdot d\br &= \int_\pi^{3\pi}\left(\cos^3(t)-3\sin^2(t)\right)\,dt\\
    &= \int_\pi^{3\pi}\left((1-\sin^2(t))\cos(t)-\frac32(1-\cos(2t))\right)\,dt\\
    &= \left(\sin(t)-\frac13\sin^3(t)-\frac32\left(t-\frac{\sin(2t)}{2}\right)\right)\Big|_\pi^{3\pi}\\
    &= -3\pi.
  \end{split}\]
  Note that this result can be obtained more quickly by observing that $\int_\pi^{3\pi}\cos^3(t)\,dt=0$ because of symmetry (graph it to see for yourself). Moreover, $\int_\pi^{3\pi}\sin^2(t)\,dt=\pi$ because of the fact that $\sin^2(t)+\cos^2(t)=1$. We will take advantage of these, and similar, observations whenever possible.

  For the right hand side of Green's Theorem, let $\mathcal R$ denote the region enclosed by the circle. Then
  \[\begin{split}
    \iint_{\mathcal R}\left(\frac{\partial N}{\partial x}-\frac{\partial M}{\partial y}\right)\,dA &= \iint_{\mathcal R}\left(\frac{\partial}{\partial x}(-3x)-\frac{\partial}{\partial y}(y^2)\right)\,dA\\
    &= \iint_{\mathcal R}(-3-2y)\,dA\\
    &= \iint_{\mathcal{R}}-3\,dA = -3\pi.
  \end{split}\]
  Note that $\iint_{\mathcal R}y\,dA=0$ by symmetry.
\end{pracsol}
\begin{practice}p.1103 \#3\end{practice}
\begin{pracsol}
  The curve $\mathcal C$ is the circle $x^2+y^2=1$ traversed in the counter-clockwise direction. Since $\BF(\br(t))=(\sin(2t)-2\cos(2t),3\cos(2t)-4\sin(2t))$ and $\br'(t)=(-2\sin(2t),2\cos(2t))$, $\BF(\br(t))\cdot\br'(t)=-2\sin^2(2t)+4\cos(2t)\sin(2t)+6\cos^2(2t)-8\sin(2t)\cos(2t)$. Therefore,
  \[\begin{split}
    \oint_{\mathcal C}\BF\cdot d\br &= \int_0^\pi(-2\sin^2(2t)-4\cos(2t)\sin(2t)+6\cos^2(2t))\,dt\\
    &= -\pi+\cos^2(2t)\Big|_0^\pi+3\pi=2\pi.
  \end{split}\]
  For the right hand side of Green's Theorem, let $\mathcal R$ denote the region enclosed by the circle. Then
  \[\begin{split}
    \iint_{\mathcal R}\left(\frac{\partial N}{\partial x}-\frac{\partial M}{\partial y}\right)\,dA &= \iint_{\mathcal R}\left(\frac{\partial}{\partial x}(3x-4y)-\frac{\partial}{\partial y}(y-2x)\right)\,dA\\
    &= \iint_{\mathcal{R}}(3-1)\,dA=2\pi.
  \end{split}\]
\end{pracsol}
\begin{practice}p.1103 \#9\end{practice}
\begin{pracsol}
  The curve $\mathcal C$ is the circle $x^2+y^2=1$ traversed in the counterclockwise direction. Therefore, letting $\mathcal R$ denote the region enclosed by the circle,
  \[\begin{split}
    \oint_{\mathcal C}\BF\cdot d\br &= \iint_{\mathcal R}\left(\frac{\partial}{\partial x}(x^3)-\frac{\partial}{\partial y}(-y^2)\right)\,dA\\
    &= \iint_{\mathcal R}(3x^2+2y)\,dA\\
    &= \int_0^{2\pi}\int_0^1 3r^2\cos^2(\theta)\cdot r\,dr\,d\theta\\
    &= \frac34\int_0^{2\pi}\cos^2(\theta)\,d\theta\\
    &= \frac{3\pi}{4}.
  \end{split}\]
  Note that $\iint_{\mathcal R}y\,dA=0$ by symmetry.
\end{pracsol}
\begin{practice}p.1103 \#15\end{practice}
\begin{pracsol}
  The planar region is $\mathcal R=\{(x,y):-1\leq x\leq 5,x^2\leq y\leq 4x+5\}$. Therefore, letting $\mathcal C$ denote its boundary curve,
  \[\begin{split}
    \oint_{\mathcal C}\BF\cdot d\br &= \iint_{\mathcal R}\left(\frac{\partial}{\partial x}(xy)-\frac{\partial}{\partial y}(x^2y)\right)\,dA\\
    &= \int_{-1}^5\int_{x^2}^{4x+5}(y-x^2)\,dy\,dx\\
    &= \int_{-1}^5\left(\frac{x^4}{2}-4x^3+3x^2+20x+\frac{25}{2}\right)\,dx\\
    &= \frac{648}{5}.
  \end{split}\]
\end{pracsol}
\begin{practice}p.1103 \#23\end{practice}
\begin{pracsol}
  The region is $\mathcal R=\{(x,y):0\leq y\leq 1,-y\leq x\leq y\}$. If $\mathcal C$ denotes its positively oriented boundary curve, then according to formula 15.39,
  \[\begin{split}
    \oint_{\mathcal C}\BF\cdot\bn ds &= \iint_{\mathcal R}\nabla\cdot\BF\,dA = \iint_{\mathcal R}\left(\frac{\partial M}{\partial x}+\frac{\partial N}{\partial y}\right)\,dA\\
    &= \int_0^1\int_{-y}^y y^2\,dx\,dy=\int_0^1 2y^3\,dy=\frac12.
  \end{split}\]
\end{pracsol}

\newpage

\section{Homework problems -- submit these}

\begin{problem}
  Let $\BF=2y\bi+x\bj$.
  \begin{enumerate}[(a)]
    \item Compute $\int_C\BF\cdot d\br$, where $C$ is the unit circle traversed in a counterclockwise direction.
    \begin{solution}
      $\BF$ is not conservative, so we must calculate by hand. Parametrize $C$ by $\br(t)=(\cos(t),\sin(t))$, $0\leq t\leq 2\pi$. Then
      \[\begin{split}
        \int_C\BF\cdot d\br &= \int_0^{2\pi}(2\sin(t),\cos(t))\cdot (-\sin(t),\cos(t))\,dt\\
        &= \int_0^{2\pi} (-2\sin^2(t)+\cos^2(t))\,dt\\
        &= -2\int_0^{2\pi}\sin^2(t)\,dt+\int_0^{2\pi}\cos^2(t)\,dt\\
        &= -2\pi+\pi\\
        &= -\pi.
      \end{split}\]
      At this point I've memorized that the integrals of $\sin^2(t)$ and $\cos^2(t)$ over $[0,2\pi]$ are both equal to $\pi$, which is how I went from the third line to the fourth line.
    \end{solution}
    \item Use the result part (a) to compute $\int_E \BF\cdot d\br$, where $E$ is the unit circle traversed in a clockwise direction.
    \begin{solution}
      Notice that $E=-C$ and we showed in class that
      \[\oint_{-C}\BF\cdot b\br=-\oint_C\BF\cdot b\br\]
      for any vector field $\BF$. Hence the answer is the negative of what we got for (a), so the answer is $\pi$.
    \end{solution}
    \item Use Green's theorem for parts (a) and (b) and verify you get the same answers.
    \begin{solution}
      Green's theorem says
      \[\oint_C\BF\cdot b\br = \iint_D\curl_\text{2D}\BF\,dA=\iint_D(N_x-M_y)\,dA.\]
      Here, $D$ is the unit disk and $M,N$ are the components of $\BF$. In the present scenario, $M(x,y)=2y$ and $N(x,y)=x$. Then $N_x=1$ and $M_y=2$, so $N_x-M_y=-1$, and so
      \[\oint_C\BF\cdot b\br=\iint_D (N_x-M_y)\,dA=\iint_D-\,dA=-\pi.\]
      This answers part (a) with Green's theorem. To answer part (b), Green's theorem does not directly apply to clockwise oriented line integrals. However, we can obtain a clockwise version of Green's theorem by multiplying both sides of Green's theorem by $-1$, then applying the theorem mentioned in part (b), to obtain
      \[\oint_E\BF\cdot b\br = \iint_D(M_y-N_x)\,dA.\]
      This works out to be $\pi$.
    \end{solution}
    \item What are the answers to parts (a) and (b) if $\BF$ is instead the vector field $\nabla u$, where $u(x,y)=\sum_{k=0}^{100}x^k y^{100-k}$?
    \begin{solution}
      $\BF$ is conservative (indeed, we are directly given that $\BF=\nabla u$) and we're integrating over a closed curve in both (a) and (b). So both line integrals are 0.
    \end{solution}
  \end{enumerate}
\end{problem}

\begin{problem}
  Calculate $\nabla\times(\nabla\times \BF)$ where $\BF(x,y,z)=(x^2+y^2,y^2+z^2,xy^2)$.
\end{problem}
\begin{solution}
  Two step computation. First,
  \[\begin{split}
    \nabla\times\BF &= \nabla\times(x^2+y^2,y^2+z^2,xy^2)\\
    &= ((xy^2)_y-(y^2+z^2)_z,(x^2+y^2)_z-(xy^2)_x,(y^2+z^2)_x-(x^2+y^2)_y)\\
    &= (2xy-2z,0-y^2,0-2y)\\
    &= (2xy-2z,-y^2,-2y).
  \end{split}\]
  Finally,
  \[\begin{split}
    \nabla\times(\nabla\times\BF) &= \nabla\times (2xy-2z,-y^2,-2y)\\
    &= ((-2y)_y-(-y^2)_z,(2xy-2z)_z-(-2y)_x,(-y^2)_x-(2xy-2z)_y)\\
    &= (-2-0,-2-0,0-2x)\\
    &= (-2,-2,-2x).
  \end{split}\]
\end{solution}

\begin{problem}
  Prove that the identity
  \[\nabla\times(g\nabla h+h\nabla g)=\mathbf 0\]
  holds for all twice continuously differentiable functions $g(x,y)$ and $h(x,y)$.
\end{problem}
\begin{solution}
  Here's a slick solution.

  \textbf{Claim}. $g\nabla h+h\nabla g=\nabla(gh)$.
  \begin{proof}
    We calculate
    \[\begin{split}
      \nabla(gh) &= \nabla(g(x,y)h(x,y)) = \left((gh)_x,(gh)_y\right)\\
      &= \left(gh_x+hg_x,gh_y+hg_y\right)\\
      &= g(h_x,h_y)+h(g_x,g_y)\\
      &= g\nabla h+h\nabla g.
    \end{split}\]
    In the second line we used the product rule for partial derivatives. It works just like the usual product rule.

    To conclude, we note that
    \[\nabla\times(\nabla(gh))=0\]
    because this is a curl of a gradient of something, which is zero -- all gradient vector fields (aka conservative vector fields) are irrotational/closed.

    See also the grad-curl-div theorem (see p.1091 Theorem 1(c)) for the same statement.
  \end{proof}
\end{solution}

\begin{problem}
  Blast from the past! One of the polar area integration examples we did in class was finding the area of the cardioid $r=3(1+\cos\theta)$, which we found to be $\frac{27\pi}2$. Use (13.5.4) to find the same area via a line integral.
\end{problem}
\begin{solution}
  The cardioid can be parametrized by
  \[\br(t) = 3((1+\cos t)\cos t,(1+\cos t)\sin t),\ 0\leq t\leq 2\pi.\]
  The derivative $\br'(t)$ is then
  \[\begin{split}
    \br'(t) &=3(-\sin t\cos t+(1+\cos t)(-\sin t),-\sin t\sin t+(1+\cos t)\cos t)\\
    &= 3(-\sin t-2\sin t\cos t,\cos t+\cos^2(t)-\sin^2(t)).
    % &= 3(-\sin t-\sin(2t),\cos t+\cos(2t)).
  \end{split}\]
  Equation (13.5.4) gives
  \[\begin{split}
    A &= \frac12\oint_C(-y\,dx+x\,dy)\\
    &= \frac12\int_0^{2\pi}(-y(t)x'(t)+x(t)y'(t))\,dt\\
    &= \frac{3\cdot 3}2\int_0^{2\pi}(-(1+\cos t)(\sin t)(-\sin t-2\sin t\cos t)\\
    &\hspace{2cm}+(1+\cos t)(\cos t)(\cos t+\cos^2t-\sin^2t))\,dt\\
    &= \frac92\int_0^{2\pi}(1+\cos t)(\sin^2t+(\sin t)(2\sin t\cos t)\\
    &\hspace{3cm}+\cos^2 t+(\cos t)(\cos^2 t-\sin^2 t))\,dt\\
    &= \frac92\int_0^{2\pi}(1+\cos t)(1+2\sin^2 t\cos t+(\cos t)(1-2\sin^2 t))\,dt\\
    &= \frac92\int_0^{2\pi}(1+\cos t)^2\,dt\\
    &= \frac92\int_0^{2\pi}(1+2\cos t+\cos^2 t)\,dt.
  \end{split}\]
  In the fifth line we used the trig identity $\cos^2t-\sin^2t=\cos(2t)=1-2\sin^2t$.
  The integral of 1 is $2\pi$, the integral of $2\cos t$ is 0, and the integral of $\cos^2 t$ is $\pi$. Therefore, the integral evaluates to
  \[\frac92\cdot (2\pi+0+\pi)=\frac{27\pi}2.\]
\end{solution}

\begin{problem}
  How difficult was each problem? Rate each problem (and part) on a difficulty scale from 1 to 7, where 1 means ``super easy, barely an inconvenience!'' and 7 means ``hardest problem I've ever done.''
\end{problem}

\newpage

\section{Hints}
\begin{hint}[Hint for 1(a)]
  The answer is the answer to (a) is $-\pi$.
\end{hint}

\begin{hint}[Hint for 3]
  Theorem 1c on p.1091 may be useful.
\end{hint}

\begin{hint}[Hint for 4]
  You will need to parametrize a polar curve into something of the form $\br(t)$. It's easier than you think. Follow your nose.
\end{hint}
